%---------------
%╔═╗╔═╗╔╦╗╦ ╦╔═╗
%╚═╗║╣  ║ ║ ║╠═╝
%╚═╝╚═╝ ╩ ╚═╝╩  
%---------------

% language setup
\newcommand{\docLanguage}{ngerman}
%\newcommand{\docLanguage}{english}

% DOCUMENT SETUP
\documentclass[12pt, oneside, a4paper, \docLanguage]{report}
\usepackage[left=3cm, 
			right=2.5cm, 
			top=2.5cm, 
			bottom=2.5cm, 
			includehead, 
			includefoot]{geometry}

% line spacing
\usepackage{setspace}
\setstretch{1,25} % 15/12 --> 1.25

% encoding setup
% T1 font encoding for languages that use a latin alphabet
\usepackage[T1]{fontenc} 

% enhanced input encoding handling - utf8 for äÄüÜöÖß...
\usepackage[utf8]{inputenc}

%de­fines Adobe Times Ro­man as de­fault text font
\usepackage{mathptmx}
\usepackage{times} % needed for acronym package

%PDF linking package
\usepackage[hidelinks]{hyperref}

% Language Setup
\usepackage[\docLanguage]{babel}
% after babel - set chapter string
\AtBeginDocument{\renewcommand{\chaptername}{}}

% language specific bibliography style
\usepackage[numbers, square]{natbib}
%\setcitestyle{square,aysep={},yysep={;}}
\usepackage[fixlanguage]{babelbib}
\selectbiblanguage{\docLanguage}
% bliographystyle setup
% babel specific: babplain, babplai3, babalpha, babunsrt, bababbrv, bababbr3
\bibliographystyle{babunsrt}


% enumeration
\usepackage{enumitem}
% tabular extension tabularx
\usepackage{tabularx}

% math packages
\usepackage{amsmath}
\usepackage{nicefrac}
\usepackage{amsthm}
\usepackage{amsbsy}
\usepackage{amssymb}
\usepackage{amsfonts}
%\usepackage{MnSymbol}


%special characters
\usepackage{amssymb}
\usepackage{upgreek,textgreek}

% acronym package
\usepackage[printonlyused, footnote]{acronym}

% breakable text in \seqsplit{}
\usepackage{seqsplit}

% \textmu
\usepackage{textcomp}

% package provides a way to compile sections of a document using the same preamble as the main document
\usepackage{subfiles}

% driver-independent color extension - used by listings,tabularx
\usepackage[usenames,dvipsnames,table,xcdraw]{xcolor}

% -- SYNTAX HIGHLIGHTING --
\usepackage{listings}
%\input{cfgs/listings/listings_def_lang_bash-cmd.tex} % adds style BASH_CMD
%\input{cfgs/listings/listings_def_lang_bash-script.tex} % adds style BASH_SCRIPT
\input{cfgs/listings/listings_def_lang_latex.tex} % adds style LATEX
%\input{cfgs/listings/listings_def_lang_matlab.tex} % adds style MATLAB
\input{cfgs/listings/listings_def_lang_python.tex} % adds style PYTHON
%\input{cfgs/listings/listings_def_lang_c++.tex} % adds style CPP
%\input{cfgs/listings/listings_def_lang_c.tex} % adds style C
%\input{cfgs/listings/listings_def_lang_json.tex} % adds style JSON

% HEADLINE CFG
\usepackage{fancyhdr} % Headers and footers
\usepackage{lastpage}
\usepackage{ifthen}
\setlength{\headheight}{1.5cm} 
%\pagestyle{fancy} % All pages have headers and footers
% override plain page style for \part, \chapter or 
% \maketitle, which implicit specifies plain page style
\input{cfgs/fancyhdr/fancyhdr_pagestyle_plain.tex}
% set list pagestyle
\input{cfgs/fancyhdr/fancyhdr_pagestyle_preface.tex}
% set default pagestyle
\input{cfgs/fancyhdr/fancyhdr_pagestyle_default_onepage.tex}
%\input{cfgs/fancyhdr/fancyhdr_pagestyle_default_twopage.tex}

\renewcommand{\chaptermark}[1]{\markright{#1}{}}
\renewcommand{\sectionmark}[1]{\markright{#1}{}}
\renewcommand{\headrulewidth}{0pt}
\renewcommand{\footrulewidth}{0pt}

% PICTURE CFG 
\usepackage{verbatim}
\usepackage{graphicx}
\usepackage{epstopdf}
\usepackage{caption}
\usepackage[list=true,listformat=simple]{subcaption}
% floating prevention packages
\usepackage{float}    % used with [H] positioning parameter
\usepackage{placeins} % \FloatBarrier 
% tikz packages
\usepackage{tikz}
\usepackage{standalone}
\usepackage{pgfplots}

% csv 
\usepackage{csvsimple}

% include only specified tex files - uncommend here
\includeonly{preface/cover,
             preface/abstract,
             preface/tableofcontents,
             preface/listoffigures,
             preface/listoftables,
             preface/lstlistoflistings,
             appendix/bibliography}

%-------------------
%╔═╗╔╦╗╦═╗╦╔╗╔╔═╗╔═╗
%╚═╗ ║ ╠╦╝║║║║║ ╦╚═╗
%╚═╝ ╩ ╩╚═╩╝╚╝╚═╝╚═╝
%-------------------
\newcommand{\strLecture}{Signale, Systeme und Sensoren}
\newcommand{\strDate}{\today}
\newcommand{\strAuthorA}{Tim Koehler}
\newcommand{\strAuthorB}{Roland Burke}
%\newcommand{\strAuthorC}{C. Author}
\newcommand{\strAuthorAEmail}{tim.koehler@htwg-konstanz.de}
\newcommand{\strAuthorBEmail}{roland.burke@htwg-konstanz.de}
%\newcommand{\strAuthorCEmail}{cauthor@htwg-konstanz.de}
% Versuchsbeschreibung 
\newcommand{\strTopic}{Aufbau eines einfachen Spracherkenners}
\newcommand{\strAbstract}{\newline
In diesem Versuch wird ein einfacher Spracherkenner aufgebaut, der z.B. zur Steuerung eines
Staplers in einem Hochregallager dienen könnte. Es reichen hierzu die Erkennung der vier einfachen Befehle ”Hoch”, ”Tief”, ”Links” und ”Rechts”
\newline
\newline
Aufgaben in diesem Versuch: 

\begin{enumerate}
\item \textbf{Fourieranalyse lang andauernder Signale}\newline
Aufnahme eines beliebigen Wortes mit einem Mikrofon. Durch eine sogenannte Triggerfunktion wird das Signal erst dargestellt
wenn eine vorgegebene Schwelle überschritten wird. Anschließend wird das Amplitudensprktrum des Signals dargestellt.\newline
Im letzten Abschnitt des ersten Teils wird dann die Windowing Methode implementiert.
\item \textbf{Spracherkennung}\newline
Der zweite Teil beschäftigt sich mit der Entwicklung eines Spracherkenners.\newline
Dazu werden einige Aufnahmen der oben gennenaten Worte gemacht und daraus Prototypen erstellt.
Anschließend wird durch Korrelation der Spektren bestimmt welches der Prototypen Worte am besten zu dem Test Signal passt.
\end{enumerate}
}
% hyperref customization
\hypersetup{
	pdftitle     = {\strTopic}, % title
	pdfsubject   = {\strLecture}, % subject of the document
	pdfauthor    = {\strAuthorA, \strAuthorB}, % author
	pdfkeywords  = {}, % list of keywords
	pdfcreator   = {}, % creator of the document
	pdfproducer  = {}, % producer of the document
	colorlinks   = false, % false: boxed links; true: colored links
	linkcolor    = red, % color of internal links (change box color with linkbordercolor)
    citecolor    = green, % color of links to bibliography
    filecolor    = magenta, % color of file links
    urlcolor     = cyan, % color of external links
	%bookmarks    = true, % show bookmarks bar?
	unicode	     = true, % non-Latin characters in Acrobat’s bookmarks
	pdftoolbar   = true, % show Acrobat’s toolbar?
	pdfmenubar   = true, % show Acrobat’s menu?
    pdffitwindow = false, % window fit to page when opened
	pdfnewwindow = true % links in new PDF window
}

%-----------------------------------------
% ╔╗ ╔═╗╔═╗╦╔╗╔  ╔╦╗╔═╗╔═╗╦ ╦╔╦╗╔═╗╔╗╔╔╦╗ 
% ╠╩╗║╣ ║ ╦║║║║   ║║║ ║║  ║ ║║║║║╣ ║║║ ║  
% ╚═╝╚═╝╚═╝╩╝╚╝  ═╩╝╚═╝╚═╝╚═╝╩ ╩╚═╝╝╚╝ ╩  
%-----------------------------------------

\begin{document}
\pagenumbering{Roman} 

\setcounter{section}{0}
\include{preface/cover}

\include{preface/abstract}
\clearpage

%
% TABLE OF CONTENTS
%
\include{preface/tableofcontents}

%
% Abbildungsverzeichnis
%
\include{preface/listoffigures}

%
% Tabellenverzeichnis
%
%\include{preface/listoftables}

%
% Listingverzeichnis
%
\include{preface/lstlistoflistings}


%--------------------------
% ╔═╗╦ ╦╔═╗╔═╗╔╦╗╔═╗╦═╗╔═╗ 
% ║  ╠═╣╠═╣╠═╝ ║ ║╣ ╠╦╝╚═╗ 
% ╚═╝╩ ╩╩ ╩╩   ╩ ╚═╝╩╚═╚═╝ 
%--------------------------

\pagenumbering{arabic}
\setcounter{page}{1}
\pagestyle{default}
%
% CHAPTER Einleitung
%
\chapter{Einleitung}
\label{chap:EINL}
\begin{normalsize}
Wie oben bereits erwähnt wird in diesem Versuch ein einfacher Spracherkenner entwickelt, der nur simple Befehle unterscheiden kann.
Der Aufbau des Spracherkenners folgt
dem in der Vorlesung beschriebenen Prinzip des Prototyp-Klassifikators. Die zugehörigen
Spektren werden mit der Windowing-Methode berechnet.\newline
Für beide Aufgaben wird eine Mikrofon verwendet, das über einen Klinkenstecker direkt
mit der Soundkarte des Computers verbunden ist. Python stellt über das Paket PyAudio ein
bequemes Interface zum Auslesen der Soundkarte zur Verfügung.\newline
Die Dokumentation dieses gesamten Experimentes ist wie immer im folgenden Erläutert.
\end{normalsize}

%
% CHAPTER Versuch 1
%
\chapter{Versuch 1}
\label{chap:VERSUCH_1}

\section{Fragestellung, Messprinzip, Aufbau, Messmittel}
\label{chap:VERSUCH_1_FRAGESTELLUNG}
Das Experiment wird begonnen indem ein beliebiges Sprachsignal mit einem Mikrofon aufgenommen wird.
Mit Hilfe einer Triggerfunktion soll das Signal so zugeschnitten werden, dass das Sample mit dem ersten Buchstaben beginnt.
Im Anschluss wird dieses Signal dann in sein Spektrum zerlegt und dargestellt.
Als letztes soll noch die Windowing Methode implementiert werden um das Signal genauer zu analysieren.
Der Aufbau besteht also lediglich aus einem Mikrofon und einem Kabel zur verbindung mit dem Eingang des Computers.

\section{Messwerte}
\label{chap:VERSUCH_1_MESSWERTE}
Das in unserem Fall gewählte Sprachsignal lautet: 'Spaghetti Eis':
\begin{figure}[H]
	\centering
	\includegraphics[width=1\textwidth]{../Images/SpaghettiEis.png}
	\caption{Audiosignal 'Spaghetti Eis' ohne Trigger}
\end{figure}
	
	
\section{Auswertung}
\label{chap:VERSUCH_1_AUSWERTUNG}
Hier is das selbe Signal wie zuvor zu sehen nur dass es dieses mal mit Triggerfunktion aufgenommen wurde:
\begin{figure}[H]
	\includegraphics[width=1\textwidth]{../Images/SpaghettiEisCuted.png}
	\caption{Audiosignal 'Spaghetti Eis' mit Trigger}
\end{figure}
Damit das Signal nach dem abschneiden trotzdem eine Sekunde lang bleibt, 
wird mit Hilfe von ein paar Zeilen Python Code
das restliche Array mit Nullen gefüllt.
\newline
\newline
Die nun folgende Grafik zeigt das zugeschnittene Signal von oben in seinem Amplitudenspektrum:
\begin{figure}[H]
	\includegraphics[width=1\textwidth]{../Images/SpaghettiEisFFT.png}
	\caption{Amplitudenspektrum des 'Spaghetti Eis' Signals mit Trigger}
\end{figure}
Das unten stehende Schaubild zeigt das Spektrum welches mit der Windowing Methode transformiert wurde.\newline
Dazu haben wir das Signal in Abschnitte mit einer Länge von 512 Samples, die sich jeweils zur
Hälfte überlappen, zerlegt. Jedes Fenster wurde mit einer Gaußschen
Fensterfunktion gewichtet, die so gewählt wird, dass die Fensterbreite 4 Standardabweichungen
entspricht. In jedem Fenster haben wir eine lokale Fouriertransformation durchgeführt und
dann die Fouriertransformierte über alle Fenster gemittelt.

\begin{figure}[H]
	\includegraphics[width=1\textwidth]{../Images/SpaghettiEisWindowed.png}
	\caption{Spektrum des 'Spaghetti Eis' Signals mit Windowing}
\end{figure}

\section{Interpretation}
\label{chap:VERSUCH_1_INTERPRETATION}
In dem Schaubild 2.2 gibt es nicht viel zu interpretieren. Hier ist nur sehr gut zu sehen wie das Signal auf den 
beginn des ersten Wortes durch die Triggerfunktion zugeschnitten wurde.\newline
Viel interessanter sind hier die beiden Spektren. Zum einen haben wir hier das welches mit dem einfachen FFT transformiert wurde
und zum anderen das Spektrum welches mit einem Gauß als Windowing Funktion bestimmt wurde.
Es ist schön zu erkennen wie viel 'saberer' das zweite Spektrum berechnet wurde. Die Grundfrequenz und ihre folgenden Harmonischen
sind weiterhin gleich aber diese sind deutlich besser zu erkennen.

%
% CHAPTER Versuch 2
%
\chapter{Versuch 2}
\label{chap:VERSUCH_2}

\section{Fragestellung, Messprinzip, Aufbau, Messmittel}
\label{chap:VERSUCH_2_FRAGESTELLUNG}
\begin{normalsize}
Im 2. Teil des Versuchs soll ein Spracherkenner gebaut werden, der vier verschiedene Worte erkennt. Dafür wird jedes Wort 5 mal aufgenommen,
und ein Prototyp ausgerechnet indem von allen 5 die Spektren berechnet werden und anschließend der Mittelwert. Zum Vergleich der Prototypen mit einem
anderen Signal wird der Korrelationskoeffizient nach Bravais-Pearson berechnet.
Dafür verwenden wir ein Mikrofon und Python.
\end{normalsize}

\pagebreak

\section{Messwerte}
\label{chap:VERSUCH_2_MESSWERTE}
\begin{normalsize}
Im folgenden sind die Signale der der vier Wörter hoch, runter, links, rechts geplottet:
\begin{figure}[H]
\centering
	\includegraphics[width=0.8\textwidth]{../Images/hoch_signal.png}
	\caption{Audiosignal vom Referenz Wort "hoch"}
\end{figure}

\begin{figure}[H]
\centering
	\includegraphics[width=0.8\textwidth]{../Images/runter_signal.png}
	\caption{Audiosignal vom Referenz Wort "runter"}
\end{figure}

\begin{figure}[H]
\centering
	\includegraphics[width=0.8\textwidth]{../Images/links_signal.png}
	\caption{Audiosignal vom Referenz Wort "links"}
\end{figure}

\begin{figure}[H]
\centering
	\includegraphics[width=0.8\textwidth]{../Images/rechts_signal.png}
	\caption{Audiosignal vom Referenz Wort "rechts"}
\end{figure}
\end{normalsize}

\pagebreak

\section{Auswertung}
\label{chap:VERSUCH_2_AUSWERTUNG}
\begin{normalsize}
Für den Spracherkenner brauchen wir zunächst die Referenzspektren für die vier Befehle
”Hoch”, ”Tief”, ”Links” und ”Rechts”. Dazu haben wir jeweils deren Spektren mit der Windowing-Methode aus Aufgabe 1 berechnet.
Die 5 Beispiele stammen vom selben Sprecher. Das Referenzspektrum haben wir durch Mittelung aller 5 Spektren erhalten.
\begin{figure}[H]
\centering
	\includegraphics[width=0.8\textwidth]{../Images/hoch_proto.png}
	\caption{Prototyp "hoch"}
\end{figure}

\begin{figure}[H]
\centering
	\includegraphics[width=0.8\textwidth]{../Images/runter_proto.png}
	\caption{Prototyp "runter"}
\end{figure}

\begin{figure}[H]
\centering
	\includegraphics[width=0.8\textwidth]{../Images/links_proto.png}
	\caption{Prototyp "links"}
\end{figure}

\begin{figure}[H]
\centering
	\includegraphics[width=0.8\textwidth]{../Images/rechts_proto.png}
	\caption{Prototyp "rechts"}
\end{figure}

\centering
\begin{tabular}{*{3}{c|c|c|p{2cm}}}
Wörter & Tim & Roland \\
\hline
hoch & 1/5 & 0/5 \\
runter & 1/5 & 1/5 \\
links & 3/5 & 2/5 \\
rechts & 2/5 & 0/5 \\

\end{tabular}
\captionof{table}{Erkennungsraten des Spracherkenners}
\begin{flushleft}
Die Wörter für das Referenzspektru wurden von Roland eingesprochen.
Die Hit-Rate des Spracherkenners beträgt: 10/40 Wörter = 25\%\newline
Die Fehlerrate beträgt: 30/40 Wörter = 75\%
\end{flushleft}

\begin{center}
	\begin{tabular}{ l l }
	\textbf{Korrelationskoeffizient:} & $r_{fg} = \frac{\sigma_{fg}} {\sigma_f * \sigma_g}$ \\
	\end{tabular}
\end{center}

Um den Korrelationskoeffizinente zu erhalten, teilt man die Kovarianz durch die Standardabweichungen.

\end{normalsize}

\pagebreak

\section{Interpretation}
\label{chap:VERSUCH_2_INTERPRETATION}
\begin{normalsize}
Man kann sehen, dass sich die 4 Referenzspektren von einander unterscheiden. Dies ist wichtig für die Funktion des Spracherkenners.\newline
Dennoch hat unser Spracherkenner nur eine Hit-Rate von 25\% was verschieden Ursachen haben kann. Zum einen haben wir pro Wort nur 5 Samples
aufgenommen, was nicht gerade viel ist. Für ein Gutes Ergebnis bräuchte man deutlich mehr. Zum anderen gab es viel Hintergrundrauschen.
\end{normalsize}

%
% CHAPTER Anhang
%
\renewcommand\thesection{A.\arabic{section}}
\renewcommand\thesubsection{\thesection.\arabic{subsection}}

\chapter*{Anhang}
\label{chap:APPENDIX}
\addcontentsline{toc}{chapter}{Anhang}
\setcounter{chapter}{0}
\addtocounter{chapter}{1}
\setcounter{section}{0}

\section{Quellcode}
\label{chap:APPENDIX_SOURCECODE}

\subsection{Quellcode Versuch 1}
\label{chap:APPENDIX_SOURCECODE_V1}
\lstinputlisting[style=PYTHON, frame=single, captionpos=b, caption=Code von Versuch 1 a) und b)]{../RecordSample.py}
\lstinputlisting[style=PYTHON, frame=single, captionpos=b, caption=Code von Versuch 1 c)]{../FftSample.py}
\lstinputlisting[style=PYTHON, frame=single, captionpos=b, caption=Code von Versuch 1 d)]{../windowing.py}
\lstinputlisting[style=PYTHON, frame=single, captionpos=b, caption=Code von Versuch 1 d)]{../WindowSpaghettiEis.py}
\pagebreak

\subsection{Quellcode Versuch 2}
\label{chap:APPENDIX_SOURCECODE_V2}
\lstinputlisting[style=PYTHON, frame=single, captionpos=b, caption=Code von Versuch 2]{../VoiceRecognition.py}
\lstinputlisting[style=PYTHON, frame=single, captionpos=b, caption=Code von Versuch 2]{../CalculatePrototypes.py}

%\section{Messergebnisse}
%\label{chap:APPENDIX_MEASUREMENT_SOURCE}

%
% Literaturverzeichnis
%
\include{appendix/bibliography}

\end{document}
%------------------------------------
% ╔═╗╔╗╔╔╦╗  ╔╦╗╔═╗╔═╗╦ ╦╔╦╗╔═╗╔╗╔╔╦╗
% ║╣ ║║║ ║║   ║║║ ║║  ║ ║║║║║╣ ║║║ ║ 
% ╚═╝╝╚╝═╩╝  ═╩╝╚═╝╚═╝╚═╝╩ ╩╚═╝╝╚╝ ╩ 
%------------------------------------